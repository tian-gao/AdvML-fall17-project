% !TEX encoding = UTF-8 Unicode

\chapter{Discussion}
\label{chap:discussion}

%%%%%%%%%%%%%%%%%%%%%%%%%%

\section{Tuning parameters optimization}
\label{sec:discussion:parameter}

In Sec.\,\ref{sec:implementation:parameters} we presented a pre-defined parameter list.
For simplicity they are fixed in this work,
but there is room for improvement in performance with parameter optimization.

\subsection{Loss weighting}
In the original work \cite{Gatys:2016gj} the authors discussed the trade-off
between content representation loss and style representation loss.
Different ratios will lead to various focus on content versus style,
and the user should be able to adjust the parameter to achieve a more suitable output result
based on different needs.

In this work the ratio of content loss and style loss is set to $10^{-2}$ arbitrarily,
and according to \cite{Gatys:2016gj} this ratio exerts relatively higher emphasis on content feature extraction,
therefore the output would resemble the raw input more.

The ratio choice would also be reflected on the difference convergence rate of loss function,
as is shown in Fig.\,~\ref{fig:results:vg:loss}.

\subsection{Convolutional layer selection}
Different layers can be used for feature extraction.
Usually convolutional layers and rectified linear units layers are used.

Apart from the choice of layer, number of layers can also be adjusted,
for purpose of faster iteration or more elaborate representation.

\subsection{Gradient descent parameters}
The Adam optimizer takes four parameters, which are arbitrarily set in this project.
The topic on the optimizer is beyond the scope of this work,
and the reader can refer to \cite{kingma2014adam} for more detailed specifications.


%%%%%%%%%%%%%%%%%%%%%%%%%%

\section{Run time optimization}

All output results in this work are generated with a laptop with hardware specifications listed in Sec.\,\ref{sec:intro:spec}.
Average run time of a single batch of generation is roughly 36000 seconds, or 10 hours.

From the perspective of computing power escalation, common methods include using more CPUs,
running with GPUs, implementing cloud computing services, or set up parallel computation mechanisms.

As to algorithm optimization, necessary parameter fixation and functionality elimination is required.
In exchange for higher speed, the user would need to face with higher information loss
(in both style representation and content representation)
and more coarse output results.

There have been some works on fast style transfer, among which \cite{engstrom2016faststyletransfer} is an excellent implementation.
Perceptual loss and quality improvement in fast stylization are also discussed in \cite{Johnson:2016hp, Ulyanov:2016wk}.
