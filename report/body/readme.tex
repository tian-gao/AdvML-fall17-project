% !TEX encoding = UTF-8 Unicode
\chapter{Code Manuscript}
\label{app:readme}

%%%%%%%%%%%%%%%%%%%%%%%%%%

Please refer to \url{https://github.com/tian-gao/AdvML-fall17-project} for detailed instructions.
All source code are provided.

The following table shows the required directories and files to use the system.

	\begin{table}[!htb]
	\center
	\begin{tabular}{c|l|l}
	\hline
	& \multicolumn{1}{c|}{folder or file name} & \multicolumn{1}{c}{usage} \\ \hline
	\multirow{2}{*}{folders}
		& input & put content picture and style picture inside \\
		& output & store output pictures \\ \hline
	neural network & visual\_geometry\_group.py & pre-process the trained VGG network data \\ \hline
	Tensorflow model & neural\_network.py &
		feature extraction and model training with Tensorflow \\ \hline
	system & style\_transfer.py & main function to accept arguments \\ \hline
	\multirow{5}{*}{utility}
		& utils.py & utility functions \\
		& constants.py & VGG network layers and pre-defined parameters \\
		& settings.py & file paths definition \\
		& logger.py & formatted standard screen output \\ \hline
	data & imagenet-vgg-verydeep-19.mat & VGG network data \\
	\hline
	\end{tabular}
	\caption{Project directory}
	\label{table:layers}
	\end{table}

In order to carry out style transfer, do \\
	\texttt{python style\_transfer.py --content content.jpg --style style.jpg --output output.jpg} \\
to transfer ``style'' to ``content'' and get the ``output''.


Instructions to run the Python code and download the data
are all explained in details on the GitHub repository.
